\documentclass[10pt, a4paper]{article}
\usepackage[utf8]{inputenc}
\usepackage[ngerman]{babel}
\usepackage{csquotes}
\usepackage{pdfpages}
\usepackage{graphicx}
\usepackage{verbatim}
\usepackage{geometry}
\geometry{a4paper, top=10mm, left=10mm, right=10mm, bottom=10mm,
	headsep=10mm, footskip=12mm}
\title{Assignment 12}
\author{}
\date{}

\begin{document}
\begin{center}
	Assignment 12: Team Starc Mipsdustries\\
	Dominik Baumgartner, Thomas Samy Dafir, Sophie Reischl
\end{center}	

\ \\
\textbf{Assignment 1: Funktioniert}\\
Funktionen für Shiftoperationen in MIPS eingefügt unter Verwendung von DecodeRFormat, um Informationen aus der Instruktion zu extrahieren. Implementierung von getShamt(), die den Shiftwert aus dem Immediate-Teil der Instruktion holt. Optionalen Debug output zu Funktionen hinzugefügt.
\ \\ 
\textbf{Assignment 2: Funktioniert}\\
Scanner, Parser erweitern, dass die Shiftoperatoren "<, "> erkannt werden. Dazu die entsprechenden Symbole eingefügt. Grammar um Shiftoperationen erweitert ($gr\_shiftExpression$). Präzedenz zwischen $gr\_expression$ und $gr\_simpleExpression$. Jeweils Wert links und rechts vom Operator durch Aufruf von $gr\_simpleExpression$ ausgewertet und code für die Entsprechende Shift-Fuktion generiert. 
\ \\
\textbf{Assignment 3: Funktioniert}\\
Code-Generation bereits bei Assignment 2 implementiert.
Shift-Operatoren in leftShift()- und rightShift()-Funktionen verwendet.
\\
\textbf{Assignment 4: Funktioniert}\\
Parameter in cstar() erstellt und allen untergeordneten Grammar-Funktionen pointer auf diesen Parameter übergeben. Parameter speichert aktuellen Wert des Ausdrucks, falls dieser nur aus Konstanten besteht. Parameter enthält flag: Zeigt der jeweiligen Funktion, ob beide Operanden Konstanten sind und gefoldet werden können.
\\
\textbf{Assignment 5: Funktioniert teilweise}\\
Eckigen Klammern zu Scanner hinzugefügt. Parser erweitert, damit Deklarationen und zugriffe auf Arrays erkannt werden ($gr\_statement$: Schreiben, $gr\_factor$: Lesen, $gr\_cstar$: globale Deklaration, $gr\_variable$: lokale Deklaration). Klasse ARRAY hinzugefügt. SymbolTable um Größe-Feld erweitert. Bei der Deklaration werden Einträge in der jeweiligen symbolTable mit dem entsprechenden Scope, Offset und der jeweiligen Größe erstellt. SymbolTable-Suche erweitert, damit auch Arrays gefunden werden.
\\
\textbf{Assignment 6: nicht implementiert}\\
\\
\textbf{Assignment 7: Funktioniert}\\
Neue Funktionen $gr\_struct\_dec$ für Deklaration und $gr\_struct\_def$ für Definition implementiert. Fallunterscheidung ob Deklaration oder Definition in $gr\_cstar$ und $gr\_procedure$. Klassen $STRUCT\_DEF$ für Struct-Definitionen und $STRUCT\_F$ für Struct-Felder eingeführt. Struct Definition $\rightarrow$ $STRUCT\_DEF$-SymbolTable Eintrag $\rightarrow$ für jedes Feld $STRUCT\_F$-SymbolTable-Eintrag. SymbolTable um belongsTo Eintrag erweitert, um Felder Definitionen zuordnen zu können. SymbolTable-Suche erweitert, damit auch Structs gefunden werden. Deklarierter Struct hat Klasse "VARIABLE".
\\
\textbf{Assignment 8: Funktioniert teilweise}\\
"Arrow"-Operator zu Scanner hinzugefügt. Neue Funktion $gr\_struct\_acc$ implementiert. Aufgerufen wenn in $gr\_statement$ (schreiben) oder $gr\_factor$ (lesen) ein Identifier gefolgt von "Arrow" erkannt wird. Offset für Feld aus Scope, Offset der Struct-Variable und in $STRUCT\_F$-SymbolTableEntry des jeweiligen Felds gespeichertem Offset errechnet.
Globale und Lokale Definition sowie Deklaration und Zugriff funktionieren sowohl in Selfie, wie auch in diversen Test-Files. Implementierung der SymbolTable-Einträge als Struct funktioniert nicht. 
\\
\textbf{Assignment 9: Funktioniert}\\
Scanner erweiter, sodass UND, OR und NOT erkannt werden. Neue Grammar-Funktion $gr\_boolExpression$ erstellt (UND und OR Ausdrücke haben niedrigste Präzedenz, NOT die höchste). Not in $gr\_factor$ hinzugefügt. Wird NOT erkannt, wird Flag gesetzt. Bei Rückkehr zu $gr\_boolExpression$ wird für Ausdrücke der Form (Variable OP-Symbol Variable) das Operator-Symbol invertiert, wenn die Flag gesetzt ist. Variablen und Konstanten werden bei einem Wert $>$ 0 als true, bei 0 als false gewertet und können auch invertiert werden. Branches nach jedem bool'schen Operator eingefügt $\rightarrow$ springt zum Ende des statements, der if- oder while- Bedingung, sobald die Bedingung für Lazy Evaluation erfüllt ist. T- und F- Jump Listen speichern Adressen der Instructionen, für die ein Fixup durchgeführt werden soll.
\\
\textbf{Assignment 10: Funktioniert}\\
Gemischte UND und OR Ausdrücke. $gr\_boolExpression$ ersetzt durch: $gr\_orExpression$ (niedrigste Präzedenz) und \\
$gr\_andExpression$ (zweit-niedrigste Präzedenz). Nach jedem Bool'schen Operator Branch generiert. Lazy Evaluation: Wenn aktuelles Ergebnis 1 und folgender Operator OR, dann Branch zum ende des Statements oder der if- oder while-Bedingung. Wenn aktuelles Ergebnis 0 und folgender Operator AND $\rightarrow$ Sprung zum nächsten OR.
\\
\textbf{Assignment 11: Funktioniert}\\
Neue Library Funktion "free" erstellt. Code für "free" wird, wie auch für alle anderen Library-Funktionen am Anfang ins binary geschrieben (durchgeführt durch "emitFree()"). Aufruf von free() im Code bewirkt einen Sprung zur Adresse, an der in emitFree() definierten Instruktionen stehen. Dadurch wird im Endeffekt ein SYSCALL mit der entsprechenden Nummer ausgelöst. ImplementFree() definiert Verhalten wenn ein SYSCALL mit Nummer 4042 gefunden wird. Adressen freier Speicherzellen in Free-List verwaltet. Jeder Eintrag zeigt auf den nächsten. Gespeichert an der ersten Speicherzelle des jeweiligen gefreeten Blocks im Heap. Malloc angepasst, dass für passende Speichergrößen zuerst der erste Free-List Eintrag verwendet. Passt die Größe nicht $\rightarrow$ normale Bump-Pointer-Allocation. 
\\
\textbf{Assignment 12: Funktioniert}\\
Code-Cleanup.\\
siehe Assignment$\_$12.pdf



\end{document}